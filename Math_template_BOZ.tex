\documentclass[fleqn, a4paper,12pt]{article}

\usepackage{amsmath,amssymb,amsfonts}
\usepackage{bibentry}
\usepackage{booktabs}
%\usepackage{epic}
\usepackage{graphicx}
	\graphicspath{{Graphics/}{../Problems/Graphics/}}
\usepackage{hyperref}
\usepackage{multicol}
\usepackage{lscape}
\usepackage{multirow}
\usepackage{natbib}
	\bibliographystyle{apalike}
\usepackage{pdfpages}
%\usepackage{setspace}
\usepackage{subfig}
\usepackage{supertabular}
\usepackage[margin=3cm]{geometry}
\usepackage{amsmath,amsfonts,amssymb}
\usepackage{color}


%%%%% Title
\makeatletter
\newcommand{\testheader}[7]{
	\setlength{\parindent}{0pt}
    \setlength{\fboxsep}{3mm}
     	\begin{center}
        \vfill\Huge\bfseries #2\\
        \Large\bfseries #1 \\
        \end{center}
    \begin{table} [!h] \centering\small
	\begin{tabular} {rp{0.6\linewidth}}
	\\
	Student name: & {#3} \\
	Student number: & {#4}\\
	\end{tabular} 
	\end{table}
	
    }

\makeatother

\begin{document}
\testheader
	{Operations Research\\
	Math and assignment template}
	{BOZ\,780}	
	{My name}	
	{12345678}	


\newpage
\section*{Mathematical notation}
\noindent First of all you need to define the variables and parameters in your model. I usually do this using a tabular environment:\\

\begin{tabular}{rp{0.8\linewidth}}
$c_i\triangleq$ & the given unit cost of ordering heart valves from supplier
$i$, where $i=\{1,2,3\}$ \\
$p_{ij}\triangleq$ & the given proportion of heart valves from supplier $i$
that falls in size category $j$, where $i=\{1,2,3\}$, and $j=\{1,2,3\}$. Size
$j=1$ represents large; $j=2$ medium; and $j=3$ small heart valves. \\
$r_j\triangleq$ & the given requirement (in units) for valves of size $j$,
where $j=\{1,2,3\}$\\
$x_k\triangleq$ & $\begin{cases}
                    1 & \text{if player $k$ is in the starting
                    lineup, where $k = \{1,\ldots,7\}$} \\
                    0 & \text{otherwise}
                    \end{cases}$ 
\end{tabular} \\

\vspace{12pt}
\noindent Then you need to define the objective function and constraints. I usually use the equation environment for this purpose:

\begin{equation}
min z = \sum_{i=1}^{3}{c_i x_i}\label{eq:WV4_3-4_2-01}
\end{equation}\\

\noindent When you define the constraints you can either use the equation environment and define a separate equation for every constraint or you can use the align environment (preferred option) to ensure that the constraints are properly aligned.

\subsubsection*{Equation example}

\vspace{12pt}
\begin{equation}
x_i \leq 700 \quad \forall \quad i\in\{1,2,3\} \label{eq:WV4_3-4_2-2a}
\end{equation}
\begin{equation}
\sum_{i=1}^{3}{p_{ij} x_i} \geq r_j \quad \forall \quad j\in\{1,2,3\}
\label{eq:WV4_3-4_2-3a}
\end{equation}
\begin{equation}
x_i  \geq 0 \quad \forall \quad i\in\{1,2,3\} \label{eq:WV4_3-4_2-4a}
\end{equation}

\subsubsection*{Align example (Preferred)}

\begin{align}
x_i & \leq 700 && \forall \quad i\in\{1,2,3\} \label{eq:WV4_3-4_2-2b} \\
\sum\limits_{i=1}^{3}{p_{ij} x_i} & \geq r_j && \forall \quad j\in\{1,2,3\}
\label{eq:WV4_3-4_2-3b} \\  
x_i & \geq 0 && \forall \quad i\in\{1,2,3\} \label{eq:WV4_3-4_2-4b}
\end{align}\\

\noindent Now you need to discuss the model by referring to each of the equations/constraints:

\vspace{12pt}
\noindent The objective in~\eqref{eq:WV4_3-4_2-01} minimises the total cost of purchasing
the heart valves from the three suppliers. Constraint~\eqref{eq:WV4_3-4_2-2a}
ensures that we do not exceed the available
heart valves. The three sizes are each addressed in a constraint, as formulated
in~\eqref{eq:WV4_3-4_2-3a}. Each model formulation must end with a description of the variable sign restrictions, and is presented in~\eqref{eq:WV4_3-4_2-4a}.

\section*{Other symbols}
Additional symbols that me be required are the following:
\begin{itemize}
\item Bigger brackets \big(\Big[\bigg\{\Bigg(\Bigg)\bigg\}\Big]\big)
\item Element of $\in$
\item Superscript $A^{B}$
\item Subscript $A_{B}$
\item Fraction $\frac{A}{B}$
\end{itemize}

\subsubsection*{Note}
If you include mathematical symbols within normal text, you need to enclose the symbol
within two \$ symbols. However, the \$ symbols are not required in the equation or align environments as these are mathematical environments.
\end{document}