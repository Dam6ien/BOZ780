\documentclass[a4paper,11pt,fleqn]{report}

\usepackage{acronym}
\usepackage{amsmath,amssymb,amsfonts}
\usepackage{booktabs}
\usepackage[dvipsnames]{xcolor}
\usepackage{lipsum}  
\usepackage[margin=30mm]{geometry}
\usepackage{graphicx}
	\graphicspath{
		{Graphics/}
	}
\usepackage{hyperref}
	\hypersetup{
		colorlinks=true,
		linkcolor=blue,
		filecolor=blue,
		urlcolor=blue,
		citecolor=blue
	}
\usepackage[sort&compress]{natbib}
	\bibliographystyle{apalike}
\usepackage[mark]{gitinfo2}
 \renewcommand{\gitMark}{Branch:\,\gitBranch\,@\,\gitAbbrevHash{}; Author:\,\gitAuthorName; Date:\,\gitAuthorIsoDate~\textbullet{}}
\usepackage{soul}
\usepackage{url}
\usepackage{wasysym}

% Adding comments to the document.
\newcommand{\jwj}[1]{{\color{red}{~(jwj: #1)}}}
\newcommand{\yourInitials}[1]{{\color{blue}{~(yourInitials: #1)}}}

% Adding a note block for yourself.
\newcommand{\noteToSelf}[1]{
   {\flushleft \vspace*{3mm}
   \fbox{ \parbox{0.97\textwidth}{\textbf{Note:} #1} }
   \vspace*{3mm}\\}
}


\begin{document}
%===============================================================
% Frontmatter and title page to be added manually at the end
%===============================================================
\pagenumbering{roman}
\thispagestyle{empty}
\begin{center}
{\huge Your project title}
\vspace{20mm} \\
{\Large Your name}
\vfill

A project report/dissertation/thesis in partial fulfilment of the requirements for the degree \\
\vspace{10mm}
{\Large \textsc{Baccalareus / Magister / Philosophiae Doctor (Industrial Engineering)}} \\
\vfill
%
in the \\
\vspace{20mm}
%
{\Large \textsc{Faculty of Engineering, Built Environment, and \\ 
Information Technology}}\\
%
\vspace{10mm}
{\Large\textsc{University of Pretoria}} \\
%
\vfill
%
\today (or just `November 2017')
\end{center}

\chapter*{Abstract}
\addcontentsline{toc}{chapter}{Abstract}

\hrule\medskip
\begin{tabular}{rp{0.75\linewidth}}
	\textbf{Title:} & Your project title as formally registered \\
	\textbf{Student name:} & Your full names \\
	\textbf{Student number:} & u01234567 \\
	\textbf{Supervisor:} & Prof. Some C. Person
\end{tabular}\medskip\\
\hrule \bigskip

{\flushleft \lipsum[1]}

\lipsum[2-3]

\tableofcontents
\listoffigures\addcontentsline{toc}{chapter}{List of Figures}
\listoftables\addcontentsline{toc}{chapter}{List of Tables}

\chapter*{Acronyms}
\addcontentsline{toc}{chapter}{Acronyms}
\begin{acronym}[ABCDEF]
\acro{CTD}{Centre for Transport Development}
\acro{UP}{University of Pretoria}
\end{acronym}

\chapter{Introduction}
\pagenumbering{arabic}
\setcounter{page}{1}
\acresetall
Consultancy firms typically sell services, a product with an instantaneous expiry date. Traditional approaches of inventory managed assume that a product can be held in inventory to satisfy a later demand, an approach that does not hold when you're selling time. To combat this, consultancy firms must ensure that their resources are running optimally. However, if resources do not have work then their utilisation will be low. To combat this, it is proposed to create a new product at a discounted rate to entice customers to purchase additional work - a concept used in airlines for ticket sales.	
Pricing considerations.



A company has i projects that are scheduled for time t. There is a resource pool that may do these projects. Each resource can take a max of 8 hours per day for a project. Projects have a resource requirement. Discounted project must be scheduled to maximise the utilisation of resources over certain time period.

Assume projects are certain. Assume project created will be bought and implemented.

\section{Background}

There is usually some introductory text before you just jump in with \textbackslash\texttt{section\{\ldots\}} and \textbackslash\texttt{subsection\{\ldots\}} commands.

\subsection{Subsection}
You have to ensure that there is proper flow through your document. 
One suggestion is to plan your document by only adding the various section and subsection headings. 
You should have flow through them. 
And once there is logical flow at the structural level, you can start populating the different sections.

In \LaTeX, when you want to start with a new paragraph, you simply leave open a line in the source \texttt{$\star$.tex} file, without adding specific line break commands `\textbackslash\textbackslash'. 
The formatting of the paragraphs so that each subsequent paragraph starts with the first line indented, and \emph{no open space between lines}, is sorted out by \LaTeX~during compilation.


\chapter{Literature review}
\acresetall
So at the start of a new chapter, the first use of the acronym \ac{UP} should be written in full again.
I doubt \citet{ar:Manson2006} will think this is a rigorous review.


\chapter{Model}
\acresetall
And here is some mathematical formula expressed in~\eqref{eq:example}.
\begin{align}
y & = mx + c \label{eq:example}
\end{align}
where $m$ is the gradient of the line.

\noteToSelf{If you want to create a note for yourself, for example during the preparation of your preliminary report, on what must still be done, then you can use this \textbackslash\texttt{noteToSelf\{\ldots\}} command we programmed for you.}
\jwj{Alternatively, there is a command in the preamble of this template that allows you to create a custom comment command, just replace the \texttt{yourInitials} with, you guessed it, your own initials}



\section{Preliminary model}

\begin{tabular}{ll}
$x_{it}\triangleq$ & $\begin{cases}
                    1 & \text{if project $i$ is underway during time $t$}\\
                    0 & \text{otherwise}
                    \end{cases}$ \\
                    \\[-1em]
$y_{jt}\triangleq$ & $\begin{cases}
                    1 & \text{if improvement project $j$ is underway during time $t$}\\
                    0 & \text{otherwise}
                    \end{cases}$ \\
                    \\[-1em]
$q_t$ & resource availability at time $t$ \\
$r_i$ & resource requirement for project $i$ \\
$r_j$ & resource requirement for improvement project $j$ \\
$\tilde{z}_{it}$ & probability of project $i$ occurring at time $t$\\
\end{tabular}

\vspace{12pt}
The parameter $\tilde{z}_{it}$ can be expressed as random variable and is derrived from the stage a customer is in the sales funnel. Projects that are ongoing or started are indicated as 1. We can include a cost of normal project and the cost of a improvement project as different.
\begin{equation}
	\max z = \sum_{i\in I} x_{it}r_i\tilde{z}_{it}+ \sum_{j \in J}y_{jt}r_i
\end{equation}

Subject to

\begin{align}
	\sum_{i\in I} x_{it}r_i\tilde{z}_{it}+ \sum_{j\in J} y_{jt}r_j \leq q_t && \forall t \in T
\end{align}

\chapter{Results and discussion}
\acresetall

\chapter{Conclusion}
\acresetall
This is the chapter where you add \emph{concluding} remarks. It is not a summary.


\bibliography{Example}


\appendix
\chapter{Some data as appendix}

\end{document}
