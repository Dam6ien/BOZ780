\documentclass[a4paper,11pt]{article}

\usepackage{amsmath,amssymb,amsfonts}
\usepackage{booktabs}
\usepackage[dvipsnames]{xcolor}
\usepackage[margin=30mm]{geometry}
\usepackage{graphicx}
	\graphicspath{
		{graphics/}
	}
\usepackage{hyperref}
	\hypersetup{
		colorlinks=true,
		linkcolor=blue,
		filecolor=blue,
		urlcolor=blue,
		citecolor=blue
	}
%\usepackage[sort&compress]{natbib}
%	\bibliographystyle{apalike}
\usepackage{soul}
\usepackage{url}


% Damien added packages
\usepackage{listings}
\lstset{language=R,
    basicstyle=\small\ttfamily,
    stringstyle=\color{DarkGreen},
    otherkeywords={0,1,2,3,4,5,6,7,8,9},
    morekeywords={TRUE,FALSE},
    deletekeywords={data,frame,length,as,character},
    keywordstyle=\color{blue},
    commentstyle=\color{DarkGreen},
}
\usepackage{float}
\usepackage{subfig}

\newcommand{\jwj}[1]{{\color{red}{~(jwj: #1)}}}
\newcommand{\yourInitials}[1]{{\color{blue}{~(yourInitials: #1)}}}

\usepackage{titlesec}
\titleformat{\section}{\normalfont\large\bfseries}{}{0pt}{}
\titleformat{\subsection}{\normalfont\large\bfseries}{}{10pt}{}
\titleformat{\subsubsection}{\normalfont\small\bfseries}{}{30pt}{}


%=======================================

%Note: Include constraints that ensure if i=1 is selected, all t's in i=1 must be activated too. If i is on then all t's must be on too.

\title{BOZ780 Assignment 1}
\author{DS de Gouveia \\ 15003079}
\date{\today}

\begin{document}
\maketitle
\tableofcontents
\newpage

%=======================================================================
\section{Question 1}









%=======================================================================
\section{Question 2 - Number of crimes}

Each precinct has rate of improvement, that is current crime given number of patrol cars less future crime with an additional patrol car. The highest reduction in crime is seen as the better solution. For five planning stages, a single additional patrol car is assigned to the precinct with the greatest improvement. This process continues until the termination of the planning stages which implies that a maximum of five cars have been allocated.

\vspace{12pt}

\begin{tabular}{rl}
  $f_s(n_{i}) \triangleq$&  crime reduction rate for all precincts in planning state $s$ for $s,s+1,\dots, 5$\\
  $n_{i} \triangleq$ & number of patrol cars assigned to precinct $i$, where $i = \{1,2,3\}$  \\ 
  $c_{is}(n_i) \triangleq$ & number of reported crimes for precinct $i$ during planning stage $s$
\end{tabular}

\vspace{12pt}

The recursion function can be modelled as

\begin{align}
	f_0(n_i) &= 0 \\
	f_s(n_i) &= \max_i \{c_{is-1}(n_i) - c_{is}(n_i)\}
\end{align}

where $f_s(n_i)$ attempts to identify the maximum difference between crime rates when an additional patrol car is added.


\begin{table}[h]
\centering
\caption{Difference in crime rates per number of patrol cars.}
\begin{tabular}{c|ccc}
\hline
\textbf{n} & \textbf{Precinct 1} & \textbf{Precinct 2} & \textbf{Precinct 3} \\
\hline
0 &	0&	0&	0 \\
1&	4&	6&	6 \\
2&	3&	3&	3\\
3&	3&	2&	3\\
4&	3&	2&	2\\
5&	1&	1&	1\\
\hline
\end{tabular}
\end{table}

\begin{align}
	f_0\begin{bmatrix}
		0 \\ 0 \\ 0
	\end{bmatrix} &= \max 
	\begin{Bmatrix}
		\text{Precint 1}= & 0 \\
		\text{Precint 2}= & 0 \\
		\text{Precint 3}= & 0
	\end{Bmatrix}	
\end{align}

\begin{align}
	f_1\begin{bmatrix}
		1 \\ 1 \\ 1
	\end{bmatrix} &= \max 
	\begin{Bmatrix}
		\text{Precint 1}= & 4 \\
		\textbf{Precint 2}= & 6 \\
		\text{Precint 3}= & 6
	\end{Bmatrix}	
\end{align}

\begin{align}
	f_2\begin{bmatrix}
		1 \\ 2 \\ 1
	\end{bmatrix} &= \max 
	\begin{Bmatrix}
		\text{Precint 1}= & 4 \\
		\text{Precint 2}= & 3 \\
		\textbf{Precint 3}= & 6
	\end{Bmatrix}	
\end{align}

\begin{align}
	f_3\begin{bmatrix}
		1 \\ 2 \\ 2
	\end{bmatrix} &= \max 
	\begin{Bmatrix}
		\textbf{Precint 1}= & 4 \\
		\text{Precint 2}= & 3 \\
		\text{Precint 3}= & 3
	\end{Bmatrix}	
\end{align}

\begin{align}
	f_4\begin{bmatrix}
		2 \\ 2 \\ 2
	\end{bmatrix} &= \max 
	\begin{Bmatrix}
		\textbf{Precint 1}= & 3 \\
		\text{Precint 2}= & 3 \\
		\text{Precint 3}= & 3
	\end{Bmatrix}	
\end{align}

\begin{align}
	f_5\begin{bmatrix}
		3 \\ 2 \\ 2
	\end{bmatrix} &= \max 
	\begin{Bmatrix}
		\textbf{Precint 1}= & 3 \\
		\text{Precint 2}= & 3 \\
		\text{Precint 3}= & 3
	\end{Bmatrix}	
\end{align}

Final solution

\begin{table}[h]
	\centering
	\begin{tabular}{c|ccc}
	\hline
		& \textbf{Patrol cars} & \textbf{Crime} \\
		\hline
		\text{Precinct 1} &3 & 4\\
		\text{Precinct 2} &1 & 19\\
		\text{Precinct 3} &1 & 14\\
		\hline
		\textbf{Total} & 5 & 37\\
		\hline
	\end{tabular}
\end{table}

It should be noted that there are several other possibilities once all precincts have been allocated 1 patrol car. 
%=======================================================================
\section{Question 3}



%=======================================================================
\section{Question 4}



\end{document}
